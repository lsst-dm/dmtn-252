\documentclass[DM,authoryear,toc]{lsstdoc}
\input{meta}

% Package imports go here.

% Local commands go here.

%If you want glossaries
%\input{aglossary.tex}
%\makeglossaries

\title{Convention for identifying bits in a mask/flags image in FITS}

% Optional subtitle
% \setDocSubtitle{A subtitle}

\author{%
Gregory Dubois-Felsmann
}

\setDocRef{DMTN-252}
\setDocUpstreamLocation{\url{https://github.com/lsst-dm/dmtn-252}}

\date{\vcsDate}

% Optional: name of the document's curator
% \setDocCurator{The Curator of this Document}

\setDocAbstract{%
Describes a convention used by the Vera C. Rubin Observatory and other projects
for the identification of individual bits in a "flags image" or "mask image" -
an integer-valued image in which individual bit planes are assigned to represent
a set of Boolean values associated with individual pixels in an accompanying
main image.
The convention supplies a symbolic name for each bit plane, and optionally a
description string.
This convention applies to the serialization of such a "flags image" in FITS.
}

% Change history defined here.
% Order: oldest first.
% Fields: VERSION, DATE, DESCRIPTION, OWNER NAME.
% See LPM-51 for version number policy.
\setDocChangeRecord{%
  \addtohist{1}{YYYY-MM-DD}{Unreleased.}{Gregory Dubois-Felsmann}
}


\begin{document}

% Create the title page.
\maketitle
% Frequently for a technote we do not want a title page uncomment this to remove the title page and changelog.
% use \mkshorttitle to remove the extra pages

% ADD CONTENT HERE
% You can also use the \input command to include several content files.

It is a common convention in astronomical imaging pipelines to annotate the
main images\footnote{This term is used herein to mean whatever image is the
principal concern of the processing --- a raw instrumental image, a
calibrated instrumental image, a derived data product, etc.
Often the pixel values of a ``main image'' may represent calibrated or
uncalibrated fluxes, but that is not relevant to the discussion in this paper.}
being processed with ancillary pixel arrays of the same shape
as the main image, conveying additional information about each pixel.
These additional arrays can convey quantitative information such as a
variance or weight for each pixel, but it is often also useful to provide
Boolean ``flags'' on a per-pixel basis.

Such flags may provide a variety of information about the progress and
success or failure of the processing of the main image.
Common meanings of such flags range from error states conveying ``this
pixel's value in the main image is bad and should not be used for
further processing or science'' to purely informational states such as
``this pixel is part of a detected extended source''.
Multiple flags may be associated with a single pixel by bit-packing
them into an integer at designated bit positions.
Note that, because of the nature of thie representation, if compression
is used for such an image, it must in general be lossless.

In some cases this ``flags image'' or ``mask image'' (both terms are in
general use in the community) is provided as an additional extension in
a FITS serialization of the main image, while in other cases it may be
delivered as a separate file.

The data documentation for projects that provide such data will generally
provide a list of flags by bit position and purpose or interpretation.

Well-established conventions of software engineering suggest that the
specific bit positions not be ``hard-coded'' into project software as
bare constants, either as a bit position (e.g., 7) or as a mask (e.g.,
\verb|0x80|).
Symbolic names should be used instead.
This can be, and has been, done in a variety of ways on different
projects, from the use of C-style preprocessor constants to the use of
dictionary-like mechanisms.

Even if this is done, however, there is still the likelihood that the
persisted file artifacts for such ``mask images'' cannot be readily
understood by an end user without recourse to the documentation for
the table of bit number mappings to meanings.
This can be made more complicated by the likelihood that different
categories of images created by a project (e.g., single-epoch images,
coadded images, and image-like calibration data products) will use
different sets of flags and therefore have different mappings.

Furthermore, when images are displayed, it is highly desirable for
image visualization tools to be capable of displaying the flags,
for instance as semi-transparent overlays on the associated main
images, and to allow users to interrogate and manipulate the
visualization using the symbolic names for the flag bit positions,
not just their bit numbers.

Within several projects, enumerated below, it has therefore been seen
as valuable to have a convention for encoding the symbolic names for
the flag bit positions in the persisted image files.

This document describes a convention for doing so in FITS persistence,
and was written expressly in order to document this in the FITS
community's registry of conventions.
The authors would welcome recommendations for the representation of
equivalent information in other persistence formats for astronomical
images, e.g., in ASDF.
A corresponding convention for the annotation of bit positions in
integer-valued table columns would also be of value to the community.

This convention has its origins in internal data products produced
within the Sloan Digital Sky Survey (SDSS) imaging pipelines (though
to our knowledge it never appeared in publicly released SDSS images).
It has been in use in the Vera C. Rubin Observatory's imaging
pipelines since at least 20xx, and is available in the Rubin ``Data
Previews'' that were opened to the community in 2021--2022.

The convention is also in use in the publicly released data from
the Hyper-Suprime-Cam (HSC) project, and is being used by the NASA
SPHEREx space telescope (to be launched in 2025) image processing
pipelines.

The open-source Firefly astronomical data visualization package
from Caltech/IPAC, used in the NASA Infrared Science Archive (IRSA),
NASA Extragalactic Database (NED), and the NASA Exoplanet Archive,
as well as in the Rubin Science Platform, supports the display and
manipulation of ``mask images'' based on the convention presented
herein.

We are registering this convention in the hope that other projects
and missions in the international astronomy community will choose
to adopt it for their own data, and that the community will come to
support it in a wider variety of visualization tools and software
packages.

\section{Explanation of the Convention}

The principal purpose of this document is to describe a method for
including symbolic names for bit positions in the headers of an
individual integer-pixel-value image extension in FITS.
We believe this alone to be of significant value and widespread
applicability in the community.
We also describe below a convention for how to structure a
multi-extension FITS file to indicate the association of a ``mask
image'' extension, with such headers, with one or more other
image extensions, but this is of secondary importance.

The core of this convention is the representation as FITS headers
of a mapping between bit positions and short symbolic names for
those bit positions.
We also describe an optional mechanism for associating a longer
documentation string with each symbolic name.

An example of a mapping is shown in Table X.

% \begin{tabular}
X
% \end{tabular}

In this example, the convention would then lead to the inclusion
of the following headers in the FITS extension containing the
bit-packed integer image containing the flags:

\verb|MP_X    = 0|

\verb|HIERARCH MP_CROSSTALK = 1|

The convention incorporates the use of the ESO \verb|HIERARCH|
registered FITS convention,
% https://fits.gsfc.nasa.gov/registry/hierarch_keyword.html
relying on the statement in that document that ``The tokens may,
however, be longer than the 8 character limit of formal FITS
keywords'' in order to support symbolic names that are longer
than 5 characters.
The convention does not mandate the use of \verb|HIERARCH| for
\emph{all} \verb|MP_*| keywords, despite the superficial
consistency that might offer, in order to allow the present
convention to be used even in contexts in which \verb|HIERARCH|
is not supported, if the names are limited to 5 characters.

In the existing practice for the use of this convention,
e.g., in the Rubin Observatory software, the same mapping applies
to every image in the same category --- e.g., all calibrated
single-epoch images would share the same flag bit positions.
However, note that that is not a requirement of this convention,
as the entire point is to allow each such image to be understood
in isolation, based only upon the headers actually present.


\subsection{Optional Support for Documentation Strings}

\subsection{Suggestions for Implementation}

Community-facing libraries wishing to provide support for this
convention should permit users to query the status of individual
planes based on their symbolic names alone.
Note that as the name-to-bit-position mapping is image metadata,
in such a library the mapping cannot be represented by compile-time
constants (e.g., C preprocessor macros).

Image visualization software supporting this convention should
provide at least the following features:

Display the mapping itself.

Allow a user to specify, by their symbolic names, which flags
(i.e., which bit positions) should be visible in the visualization.

Display the names of the flags which are "true" (bit value 1)
for a given pixel in the same contexts in which the software
displays what the pixel value of a non-flags image is.

Support the visualization of a ``mask image'' in conjunction
with an associated main image (e.g., a flux image), including
both in the image display itself (perhaps as a semi-transparent
overlay), and in whatever UI element the tool uses to display
the ordinary pixel values (e.g., fluxes) of the main image.

\section{Formal Statement of the Convention}

This section describes the behavior of ``conforming implementations'';
i.e., software that supports either reading or writing bit-packed
Boolean flags as image pixel data and that provides behavior aligned
with this convention.
The standards-vocabulary terms ``shall'', ``should'', ``may'',
``should  not'', and ``must not'' in this section should be interpreted
as relative to that goal.

\subsection{Writing}

A conforming implementation shall support the input and/or output of
data in the FITS image

A conforming implementation \strong{must not} output negative bit position
numbers in the \verb|MP_*| headers, or bit position numbers exceeding the
available number of bits in the extension's pixel values.

A conforming implementation \strong{must not} use lossy compression on
a FITS extension containing bit-packed pixel flags.

\subsection{Reading}

A conforming implementation shall support the input of data in the FITS
image extension formats (image or compressed image) ...

A conforming implementation shall treat an extension as containing
bit-packed pixel flags if a) it is an integer image or compressed image
extension, b) it has one or more headers beginning with \verb|MP_| or
\verb|HIERARCH MP_| with FITS-header-key conformant names and non-negative
integer values, c) the integer values of such headers do not exceed the
number of bits in the integer pixel values of the extension, and d) any
compression used internal to the extension is lossless.

A conforming implementation \strong{may} process extensions in this way
even if \verb|MP_| headers are present with integer values exceeding the
bit width of tthe pixel values, but \strong{should} issue a warning in
such situations.

A conforming implementation \strong{must} issue an error if a compressed
image extension otherwise meeting the above criteria was compressed with a
lossy algorithm.


\section{Survey of Existing Practice}

\appendix
% Include all the relevant bib files.
% https://lsst-texmf.lsst.io/lsstdoc.html#bibliographies
\section{References} \label{sec:bib}
\renewcommand{\refname}{} % Suppress default Bibliography section
\bibliography{local,lsst,lsst-dm,refs_ads,refs,books}

% Make sure lsst-texmf/bin/generateAcronyms.py is in your path
\section{Acronyms} \label{sec:acronyms}
\addtocounter{table}{-1}
\begin{longtable}{p{0.145\textwidth}p{0.8\textwidth}}\hline
\textbf{Acronym} & \textbf{Description}  \\\hline

DM & Data Management \\\hline
\end{longtable}

% If you want glossary uncomment below -- comment out the two lines above
%\printglossaries





\end{document}
